\documentclass{pracalicmgr}
\usepackage{magisterka}
\begin{document}
	\section{Projekt OGLE}
	\subsection{Historia}
    Pierwotnym celem projektu OGLE (Eksperyment Mikrosoczewkowania Grawitacyjnego, ang. \textit{The Optical Gravitational Lensing Experiment}) było poszukiwanie ciemnej materii poprzez obserwację zjawisk mikrosoczewkowania grawitacyjnego. Ta metodologia poszukiwań została po raz pierwszy zaproponowana przez Paczyńskiego w publikacjach z 1986 i 1991 roku.
    
    Prawdopodobieństwo zarejestrowania mikrosoczewkowania jest niewielkie, ze względu na rzadkie występowanie oraz niepowtarzalność tego typu zjawisk. Nawet w najbardziej odpowiednich obszarach nieba prawdopodobieństwo to jest rzędu jednego zjawiska na milion obserwowanych gwiazd na rok. W związku z tym, aby otrzymać znaczącej wielkości próbkę zjawisk, potrzebna jest długoterminowa regularna obserwacja dużej liczby obiektów. Dodatkowym założeniem projektu było poszukiwanie tego typu zajść w skali lokalnej, gdy soczewkowany obiekt znajduje się w odległości rzędu kilkudziesięciu kiloparseków od ziemi. Najkorzystniejsze do prowadzenia obserwacji z takimi założeniami są najgęstsze obszary nieba - Zgrubienie Centralne Galaktyki oraz Obłoki Magellana.
   
    W pierwszej fazie projektu, rozpoczętej w 1992 roku, do prowadzenia obserwacji używano teleskopu Swope, znajdującego się w Obserwatorium Las Campanas w Chile. Do najważniejszych wyników naukowych uzyskanych w czasie jej trwania należy zarejestrowanie pierwszych przypadków mikrosoczewkowania grawitacyjnego w kierunku centrum galaktyki, w tym pierwszego mikrosoczewkowania przez gwiazdę podwójną. Ogółem, zaobserwowano kilkanaście przypadków mikrosoczewkowania. Wszystkie krzywe jasności dokładnie odpowiadały teoretycznym modelom mikrosoczewkowania, co pozwoliło potwierdzić idee Paczyńskiego, dotyczące istnienia mikrosoczewek.
    
    Projekt OGLE wszedł w swoją drugą fazę w 1996 roku. Aby usprawnić jego realizację, w obserwatorium Las Campanas zbudowano Teleskop Warszawski, wykorzystywany wyłącznie na potrzeby projektu OGLE. Dzięki temu ilość obserwowanych codziennie gwiazd wzrosła z kilku do kilkudziesięciu milionów. Do największych osiągnięć z tej fazy projektu należy poprawienie kalibracji gwiazd RR Liry i Cefeid jako świec standardowych, co przyczyniło się do rewizji skali odległości we wszechświecie. W tej fazie projektu zaobserwowano 160 zjawisk mikrosoczewkowania.
    
    W 2001 roku OGLE weszło w trzecią fazę projektu, co wiązało się z zainstalowaniem mozaikowej kamery CCD w Teleskopie Warszawskim. Ta zmiana pozwoliła na zwiększenie liczby monitorowanych regularnie gwiazd o rząd wielkości. Faza ta przyniosła liczne odkrycia planet pozasłonecznych, zaobserwowanych metodą tranzytu oraz metodą mikrosoczewkowania grawitacyjnego. Na podstawie danych z tej fazy sklasyfikowano około 500 tys. gwiazd zmiennych. Natrafiono między innymi na pierwszą cefeidę w układzie podwójnym, co po raz pierwszy umożliwiło dokładne wyznaczenie masy gwiazdy tego typu.
    
    Obecna, czwarta, faza projektu trwa od roku 2010. Zastosowany w niej detektor posiada pole widzenia wielkości 1,4 stopnia kwadratowego, a całkowity monitorowany obszar nieba, obejmujący Obłoki Magellana, centrum oraz dysk Galaktyki, pokrywa łącznie 3500 stopni kwadratowych. Pozwala to na regularne obserwacje blisko miliarda obiektów. Częstotliwość obserwacji jest zależna od pola - dla najgęstszych, najczęściej obserwowanych pól w centrum Galaktyki wynosi ona do 30 obserwacji w ciągu nocy.
    
    Jak dotychczas, w ramach OGLE odkryto ponad $17000$ zjawisk mikrosoczewkowania grawitacyjnego, co pozwala uznać projekt za bardzo owocny w kontekście jego pierwotnych założeń. Jego rezultaty wychodzą jednak daleko poza dziedzinę obserwacji mikrosoczewek grawitacyjnych. 
    
    Oprócz  wyszczególnionych wcześniej osiągnięć związanych z poszczególnymi fazami projektu, warto wymienić kolejne, często związane z obserwacjami wykonanymi na przestrzeni kilku faz. Należy do nich przegląd supernowych i innych obiektów przejściowych oraz kwazarów za Obłokami Magellana, poszukiwanie obiektów transneptunowych, oraz oszacowanie ilości ciemnej materii w postaci obiektów MACHO w halo galaktycznym. W ramach OGLE odkryto również wiele nowych klas obiektów, w tym gwiazd zmiennych okresowo, takich jak DPV (ang. \textit{Double-periodic variables}), BEP (ang. \textit{Binary evolution pulsators}), czy BLAP (ang. \textit{Blue large-amplitude pulsators}).
    
	\subsection{Dane}
    Przedmiotem tej pracy jest klasyfikacja obiektów jedynie w oparciu o dane fotometryczne z obserwacji prowadzonych w filtrze \textit{I}. Zebrane na przestrzeni ponad 20 lat trwania projektu OGLE dane  opisują obserwowaną jasność około miliarda obiektów. W zależności od pola, przerwy pomiędzy kolejnymi punktami pomiarowymi wynoszą od kilkunastu minut do kilku dni. Dla obiektów w Obłokach Magellana, krzywe blasku składają się średnio z kilkuset obserwacji, natomiast w przypadku najdłużej i najczęściej obserwowanych pól w centrum galaktyki, liczba punktów pomiarowych sięga kilkunastu tysięcy. 
\end{document}