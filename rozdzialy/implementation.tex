\documentclass{pracalicmgr}
\usepackage{magisterka}
\begin{document}
	\section{Realizacja projektu}
	\subsection{Wstępna analiza danych}
	\subsection{Implementacja systemu}
    \subsubsection{Reprezentacja obiektów}
    Do zbudowania reprezentacji obiektów wejściowych i referencyjnych na pierwszej warstwie sieci wykorzystano atrybuty wyliczone przy pomocy następujących pakietów i programów:
    \begin{itemize}
        \item Cesium-ML %\cite{cesium}
        \item VARTOOLS %\cite{vartools}
        \item VaST %\cite{vast}
        \item FATS %\cite{fats}
        \item UPSILoN %\cite{upsilon}
    \end{itemize}
    Pełna lista wykorzystanych w pierwszej wersji realizacji systemu, razem ze szczegółami dotyczącymi ich implementacji, znajdują się w dodatku \ref{appendix:features}. Następnie przy użyciu metod opisanych w Części \ref{training:repr}, reprezentacja obiektów została ograniczona do następujących atrybutów:
    
    
    Druga warstwa sieci zawiera dwa komparatory pozwalające na porównanie ogólnego kształtu sfazowanej krzywej blasku przy użyciu następujących algorytmów:
    \begin{itemize}
        \item DTW (ang. \textit{Dynamic Time Warping} - Dynamiczne odkształcanie czasu) 
        \item TWED (ang. \textit{Time Warping Edit Distance} - Odległość edycyjna z odkształcaniem czasu)
    \end{itemize} 
    Krótki opis ich działania oraz zaimplementowany kod w języku Java dostępne są odpowiednio w Dodatkach \ref{appendix:code-1} oraz \ref{appendix:code-2}.
    \subsubsection{Zbiór referencyjny}
    \subsubsection{Implementacja sieci}
    
\end{document}