\documentclass{pracalicmgr}
\usepackage{magisterka}
\begin{document}
	\section{Ewaluacja}
    \subsection{Wyzwania}
    Pierwszym poważnym problemem, napotkanym przy realizacji projektu, był brak kompletnego zbioru referencyjnego zawierającego w sobie przykładowe obiekty z wszystkich klas gwiazd zmiennych, obserwowanych aktualnie w projekcie OGLE, razem z ich właściwymi typami, oraz przykładowe obiekty nie należące do żadnej ze zdefiniowanych aktualnie klas.
    
    Mimo istnienia bardzo obszernego katalogu gwiazd zmiennych wykrytych dotychczas, nie podjęto się jeszcze skompilowania takiego zbioru, który mógłby być używany przy wszystkich przyszłych projektach wykorzystujących metody uczenia maszynowego do klasyfikacji automatycznej gwiazd zmiennych. Taki stan rzeczy wynika przede wszystkim z następujących czynników: po pierwsze, ze względu na 
    
    Zazwyczaj tego typu zbiór referencyjny tworzony jest w momencie rozpoczęcia nowego projektu, w miarę jego potrzeb. Jednak w przypadku, kiedy pracę nad tego projektem prowadzone są przez osobę bez wiedzy eksperckiej w dziedzinie, skonstruowanie tego typu zbioru może stanowić problem. W tym projekcie, zbiór referencyjny został skonstruowany w oparciu o katalog z trzeciej fazy projektu OGLE, co sprawia, że jest on częściowo nieaktualny. Ze względu na to, wyniki mogą być częściowo zaburzone - możemy oczekiwać zarówno występowania false negatives, jak i false positives, które nie 
    
    \subsection{Analiza wyników}
    
    
    Sposoby na udoskonalenie przygotowanego programu, zarówno w celu poprawy dokładności jak i przyspieszenia całego procesu, opisane zostaną w następnym punkcie.
    \subsection{Perspektywy na przyszłość}
    
\end{document}