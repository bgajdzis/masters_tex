\documentclass{pracalicmgr}
\usepackage{magisterka}
\begin{document}
	\section{Wstęp}
    Wśród setek milionów obiektów astronomicznych obserwowanych co nocy, wielkim zainteresowaniem zarówno wśród amatorów jak i profesjonalistów cieszą się gwiazdy zmienne okresowo. W skład tej grupy wchodzą między innymi obiekty takich typów jak . Ich istotność 
    
    Jednym z największych przeglądów fotometrycznych nieba, mającym na celu między innymi poszukiwanie i analizę gwiazd zmiennych okresowo, jest projekt OGLE. Ilość obserwowanych w ramach tego projektu obiektów, oraz ilość danych obserwacyjnych dostępnych dla każdego z nich, przyrastają w tempie, które nie pozwala na ręczne opracowanie i przeanalizowanie ich wszystkich przez ekspertów. Nie ulega wątpliwości zapotrzebowanie na automatyczny system analizy danych obserwacyjnych, który wspomagałby poszukiwanie gwiazd zmiennych okresowo. System taki miałby na celu zawężenie zbioru kandydatów na obiekty poszukiwanych typów, który później mógłby zostać przeanalizowany przez ekspertów w celu potwierdzenia lub odrzucenia zaproponowanej klasyfikacji. W dalszej perspektywie czasowej mógłby też prowadzić w pełni samodzielną klasyfikację obiektów.
    
    Przy projektowaniu takiego systemu największym wyzwaniem, poza samym ogromem dostępnych danych, jest ich zmienność i niespójność na wielu płaszczyznach. Dane obserwacyjnie pochodzące z różnych przeglądów nieba mogą różnić się między sobą filtrem, w którym prowadzone są obserwacje, metodami używanymi do wykonywania fotometrii poszczególnych obiektów, czasami odczytów czy charakterystykami błędów. Nawet w ramach jednego projektu, różne obszary nieba często różnią się takimi parametrami jak średnia obserwowana jasność, czy średnia wielkość błędu dla obiektów z tego obszaru. Nowe odkrycia prowadzą regularnie do powstawania nowych klas obiektów, zmian parametrów definiujących istniejące już klasy, dzielenia ich na podklasy lub scalania. Ponadto, zdarza się, że definicje klas różnią się pomiędzy różnymi katalogami gwiazd zmiennych. Wszystkie te czynniki sprawiają, że system zaprojektowany w celu automatycznej detekcji i analizy gwiazd zmiennych musi pozwalać na łatwe dodawanie nowych typów obiektów, bez konieczności przeprowadzania długotrwałego procesu analizy, kalibracji czy treningu w przypadku systemu opartego o uczenie maszynowe. Ponadto, system powinien operować w oparciu o jak najprostszy opis danych, co pozwoli na poszukiwanie i klasyfikację gwiazd zmiennych niezależnie od pochodzenia analizowanych danych obserwacyjnych. Takie 
    
    Niniejsza praca poświęcona jest opracowaniu systemu spełniającego te założenia, stosując metody uczenia maszynowego oparte na matematycznym formalizmie \textit{Sieci Komparatorów Rozmytych}. Do treningu i testowania systemu, użyte zostały dane obserwacyjne z III i IV fazy projektu OGLE. W pracy omawiane są szczegółowo następujące zagadnienia. W Części 2 przedstawiono podstawowe informacje dotyczące poszukiwania i klasyfikacji gwiazd zmiennych, istotności tego zagadnienia w astronomii oraz krótkie podsumowanie dotychczasowych prac z zakresu automatyzacji tej dziedziny. W Części 3 opisany został projekt OGLE będący źródłem danych wykorzystanych w omawianych badaniach. W Części 4 przybliżone są podstawowe zagadnienia z zakresu konstrukcji Sieci Komparatorów Rozmytych oraz ich zastosowania w uczeniu maszynowym. Część 5 przedstawia szczegóły implementacji systemu, który jest przedmiotem tej pracy, a Część 6 zawiera analizę i podsumowanie wyników jego testów oraz opis największych wyzwań związanych z jego realizacją i perspektyw na przyszłość.  
\end{document}