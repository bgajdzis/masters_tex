\documentclass{pracalicmgr}
\usepackage{magisterka}
\begin{document}
	\section{Gwiazdy Zmienne}
    \subsection{Informacje ogólne}
    \begin{defn}
        Gwiazdami zmiennymi nazywamy takie gwiazdy, których obserwowana wielkość gwiazdowa istotnie zmienia się w czasie. 
    \end{defn}
    Warto zaznaczyć, że, większość gwiazd wykazuje drobne zmiany w jasności. Przykładowo, jasność słońca fluktuuje około $0,1\%$ na przestrzeni trwającego 11 lat cyklu. Ponadto, każda gwiazda w trakcie swojego całego życia ewoluuje, przechodząc serię znaczących zmian na przestrzeni milionów lat. Aby gwiazda została uznana za zmienną, zmiany w jasności powinny być zauważalne w dostatecznie krótkiej skali czasowej.
    
    W zależności od przyczyny zmian blasku, gwiazdy zmienne dzielą się na dwie główne kategorie:
    \begin{itemize}
        \item Gwiazdy zmienne fizycznie, których jasność zmienia się na skutek procesów zachodzących w samej gwieździe lub otaczającej ją materii. Ze względu na charakter zmian, gwiazdy zmienne fizycznie dzieli się dalej na następujące grupy:\begin{itemize}
            \item gwiazdy pulsujące, w których struktura gwiazdy lub jej atmosfery ulega regularnym zmianom,
            \item gwiazdy wybuchowe, w których następuje jednorazowa lub powtarzająca się dramatyczna zmiana struktury,
            \item gwiazdy zmienne atmosferycznie, w których następują nieregularne zmiany w atmosferze.
        \end{itemize}
        \item Gwiazdy zmienne geometrycznie, których jasność zmienia się na skutek zmian orientacji gwiazdy względem obserwatora. Ze względu na charakter zmian, gwiazdy zmienne geometrycznie dzieli się dalej na następujące grupy:\begin{itemize}
            \item gwiazdy zaćmieniowe, czyli układy podwójne, w których jeden składnik okresowo przysłania drugi,
            \item gwiazdy świecące niesferycznie, w których przyczyną obserwowanych zmian jest rotacja gwiazdy,
            \item zjawiska mikrosoczewkowania grawitacyjnego, w których przyczyną pojaśnienia gwiazdy jest grawitacyjne ogniskowanie promieni świetlnych w kierunku obserwatora przez inny obiekt przechodzący w pobliżu linii widzenia.
        \end{itemize}
    \end{itemize}
    
    Innym kryterium podziału gwiazd zmiennych jest rozróżnienie na gwiazdy zmienne nieokresowo i okresowo. W tych pierwszych, zmiany obserwowanej jasności zachodzą jednorazowo, natomiast w przypadku tych drugich, powtarzają się z pewną regularnością. 
    
    Pierwszą odkrytą gwiazdą zmienną była omikron Ceti, później nazwana Mira (łac. \textit{mira} -- cudowna). Jej okresowe zanikanie zauważył w 1596 roku David Fabricius. Do 1786 znano 12 gwiazd zmiennych, w tym pierwszą gwiazdę zaćmieniową -- Algola. W późniejszych latach liczba zidentyfikowanych zmiennych wzrastała w coraz szybszym tempie, szczególnie gdy stało się możliwe badanie gwiazd przy pomocy fotografii nieba. Współcześnie gwiazdy sklasyfikowane jako zmienne liczy się w dziesiątkach tysięcy, a obiekty podejrzewane o bycie gwiazdą zmienną w setkach tysięcy.
    \subsection{Klasyfikacja}
    Podgrupy zarówno gwiazd zmiennych geometrycznie jak i fizycznie dzielą się dalej na bardziej szczegółowe kategorie. Poniżej zostaną przybliżone te, których automatyczne wykrywanie jest głównym zagadnieniem niniejszej pracy.
    \subsubsection{Cefeidy}
    Cefeidy klasyczne to wyjątkowo jasne gwiazdy pulsujące, regularnie zwiększające swoją objętość oraz temperaturę powierzchniową na skutek zmian absorpcji w warstwie składającej się z częściowo zjonizowanego helu, znajdującej się przy samej powierzchni gwiazdy. Zjawisko to nosi nazwę $\kappa$-mechanizmu. Krzywa blasku cefeidy przypomina sinusoidę, ale spadek jasności jest bardziej powolny, niż jej wzrost. Jasności absolutne cefeid mieszczą się w przedziale $-2^m$ do $-6^m$, a zakres amplitud zmian blasku wynosi od $0,1^m$ do $2^m$. Okres pulsacji wynosi od jednego do $135$ dni.
    
    Cefeidy typu drugiego są gwiazdami lżejszymi, starszymi i mniej jasnymi. Ich krzywe blasku mają ten sam charakterystyczny kształt, ale o krótszych okresach pulsacji, typowo poniżej 50 dni, i mniejszych amplitudach zmian jasności. 
\begin{figure}[!h]
    \subfloat[Cefeida Typu pierwszego]{
    \includegraphics[width=0.5\plotwidth,keepaspectratio]{placeholder} 
    \caption{Cefeida typu pierwszego}
    \label{fig:cep1}
    }

    \subfloat{
    \includegraphics[width=0.5\plotwidth,keepaspectratio]{placeholder} 
    \caption{Cefeida typu drugiego}
    \label{fig:cep1}
}
    
    \caption{Wykresy zależności obserwowanej jasności od fazy dla cefeid typu pierwszego i drugiego}
    \label{fig:ceps}
\end{figure}
    \subsubsection{RR Liry}
    Stare gwiazdy o niskiej masie, których zmiany jasności, podobnie jak w przypadku Cefeid, tłumaczone są przez $\kappa$-mechanizm. Gwiazdy tego typu mają znacznie krótsze okresy pulsacji oraz mniejsze jasności bezwzględne niż Cefeidy, przez co trudniej je zidentyfikować, za to występują znacznie częściej. Okres pulsacji gwiazdy RR Liry zazwyczaj wynosi mniej niż jeden dzień, a jasność przyjmuje wartości średnio $0.75^m$.  
    \subsubsection{Delta Scuti}
    Nazywane również Karłowatymi Cefeidami, gwiazdy Delty Scuti to gwiazdy pulsujące, z okresem od 30 minut do 8 godzin. Gwiazdy tego typu pulsują nieradialnie, co znaczy, że nie zachowują symetrii sferycznej, gdy się rozszerzają. Są, po gwiazdach typu ZZ Ceti, najczęściej występującym typem obiektów zmiennych w naszej galaktyce.
    \subsubsection{Gwiazdy zmienne o długim okresie}
    Chłodne i bardzo stare gwiazdy pulsujące, charakteryzujące się okresami pulsacji dłuższymi niż 100 dni i zmianami jasności większymi niż $1^m$ w paśmie podczerwonym. Ta klasa dzieli się na wiele różnorodnych podkategorii. Zależnie od katalogu gwiazd zmiennych, definicje dotyczące tego, jakie obiekty wchodzą w skład kategorii Gwiazd zmiennych o długim okresie różnią się. Zdarza się nawet, że definicja takiej klasy w ogóle nie występuje -- jak na przykład w GCVS. 
    
    Najbardziej popularną podkategorią tej klasy są Mirydy, nazwane od pierwszej odkrytej gwiazdy zmiennej, która również należy do tej grupy. Poza nimi, do kategorii gwiazd zmiennych o długim okresie zalicza się najczęściej zmienne półregularne, rzadziej RV Tauri i OSARG.
    \subsubsection{Układy zaćmieniowe}
    Określenia gwiazda podwójna używa się w odniesieniu do dwóch gwiazd leżących blisko siebie. Układy zaćmieniowe to takie gwiazdy podwójne, które są ze sobą związane grawitacyjnie i okrążają wspólny środek masy, a nachylenie płaszczyzny ich ruchu do linii widzenia pozwala zaobserwować zmiany jasności, powodowane przez zaćmienia. Układy zaćmieniowe odznaczają się charakterystycznym kształtem krzywej blasku, gdzie wyodrębnić można dwa typy lokalnych minimów -- nieco głębsze, gdy ciemniejszy składnik układu zasłania jaśniejszy, albo płytsze, w odwrotnej sytuacji. 
    \subsection{Istotność w astronomii}
    Identyfikacja gwiazd zmiennych okresowo i poprawna ich klasyfikacja mają w astronomii istotne znaczenie. Obiekty takie stanowią bardzo ważne narzędzia w astrofizyce gwiazdowej, gdyż często możemy wyznaczyć dla nich parametry takie jak jasność bezwzględna, masa, promień czy temperatura, co w ogólnym przypadku jest z reguły niemożliwe.

    Dla Cefeid, zarówno pierwszego, jak i drugiego typu, zachodzi ścisła zależność pomiędzy okresem a jasnością bezwzględną. Została ona zaobserwowana przez Henriettę Leavitt, pracownicę Harvard College Observatory, która w 1912 roku zauważyła, że jaśniejsze Cefeidy mają zawsze dłuższe okresy zmian jasności, niż mniej jasne gwiazdy tego typu. Podobna zależność występuje również dla gwiazd typu Delta Scuti, chociaż w tym przypadku jest ona znacznie mniej dokładnie określona, niż dla Cefeid. Z kolei wszystkie gwiazdy RR Liry przyjmują prawie takie same wartości jasności absolutnej. 
    
    Dla gwiazd należących do wszystkich powyższych typów, jesteśmy w stanie z dobrą dokładnością wyznaczyć jasność bezwzględną, a tym samym, poprzez porównanie tej wartości z jasnością obserwowaną, określić ich odległość od naszego układu słonecznego. Jest to jedna z podstawowych metod wyznaczania odległości w dużych skalach. Obiekty, których absolutna wielkość gwiazdowa jest znana, nazywa się świecami standardowymi.
    
    W przypadku układów zaćmieniowych, możliwe jest wyznaczenie stosunku mas, temperatur efektywnych czy względnych rozmiarów. W niektórych przypadkach, na przykład opierając się dodatkowo o znajomość danych spektroskopowych, można wyznaczyć bezwzględne wartości tych parametrów dla obu składników układu.
    
    Podobnie jak w opisanym wcześniej przypadku gwiazd pulsujących, dokładne wyznaczenie podstawowych parametrów opisujących układ zaćmieniowy pozwala na zmierzenie odległości danego obiektu. Przy użyciu tej metody zmierzono między innymi odległość do Wielkiego Obłoku Magellana z największą dotychczas dokładnością.      
    \subsection[Przegląd prac poświęconych automatycznej klasyfikacji]{Przegląd prac poświęconych poszukiwaniu i automatycznej klasyfikacji gwiazd zmiennych}
    W ostatniej dekadzie, dziedzina automatycznego poszukiwania i klasyfikacji obiektów zmiennych doczekała się wielu bardzo istotnych publikacji. \citep{huijse:pipeline}, \citep{huijse:ckp}    
\end{document}